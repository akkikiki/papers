\documentclass[11pt]{article}
\usepackage[letterpaper,margin=1in]{geometry}
\usepackage{amsbsy}
\usepackage{amsmath}
\usepackage{url}

\newcommand{\argmax}{\mathop{\rm arg~max}\limits}

\begin{document}

This note explains essential stuff to understand a Naive Bayes classifier trained with both labeled data and unlabeled data \cite{nigam2000text}.


\section{List of Important Terminologies}
\begin{enumerate}
 \item Q funtion of an EM algorithm is either of the following:
$$\sum_{d \in D} \sum_{c \in C} P(c|d; \theta) \log P(c, d; \theta)$$
$$\sum_{d \in D} E_{P(c|d; \theta)}[ \log P(c, d; \theta)]$$
 \item The objective function of a Naive Bayes classifier when conducting a maximum a posteriori estimation 
$$\log P(\theta) + \log P(D)$$
\item Let $q_{w,c}$ be a probability that a word $w$ is chosen given class $c$ i.e.
$$q_{w,c} = P(W = w | C = c)$$
\end{enumerate}

\section{Notes}
\begin{enumerate}
 \item Since we want to handle probabilities, the constraint $\sum_{c} p_c = 1$ is set. Therefore, we want to optimize using the method of Lagrange multiplier.
 \item Assume that the data likelihood when incorporating both labeled data and unlabeled data is
$$\log P(D^{l})P(D^{u}) =\log P(D^{l}) + \log P(D^{u})$$
As a result, the objective function of a Naive Bayes classifier becomes 
\begin{equation*}
\begin{aligned}
& \underset{\theta}{\text{maximize}}
& & \log P(\theta) + \log P(D^{l}) + \log P(D^{u}) \\
& \text{subject to}
& & \sum_{c \in C} p_c = 1 \\
& 
& & \sum_{w} q_{w,c} = \sum_{w} P(W = w | C = c) = 1
\end{aligned}
\end{equation*}
\end{enumerate}

 

\bibliographystyle{plain}
\bibliography{yoshinari_notes_em}

\end{document}

