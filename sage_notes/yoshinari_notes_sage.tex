\documentclass[11pt]{article}
\usepackage[letterpaper,margin=1in]{geometry}
\usepackage{amsbsy}
\usepackage{amsmath}
\usepackage{bm}
\usepackage{amssymb}
\usepackage{url}

\newcommand{\argmax}{\mathop{\rm arg~max}\limits}

\begin{document}

\section{Summary}

This is a note for \cite{eisenstein2011sparse}.
Let's start from Naive Bayes classifier.
$$
P(word = w|class = k) = \frac{\exp(\eta_{kw})}{\sum_v \exp(\eta_{kv})}
$$

The important part is $\eta_{kw} \in \mathbb{R}$, which allows $\eta_{kw}$ to take negative values too. If we directly model the probability $p_{kw} = \frac{\# \text{ word $w$ in class } k}{\# \text{ words in class } k}$, then it has the limitation that 1) $1 \geq p_{kw} \geq 0$ and 2) $\int_{\Omega} p_{kw} = 1$. This technique is called ``natural parameterization'' denoted as $\eta_{kw} = \log(p_{kw}) + c$ where $c$ is a constant value.
Also note that log-linear models are easy to take the derivatives.
%TODO: What is the resulting formula for taking the derivative?



$$
P(w|y_d, m, \eta) = \frac{\exp(m + \eta_{y_d})}{\sum_i (m_i + n_{y_d, i})}
$$

Now, the prior is no longer limited to Dirichlet distribution. For example, a prior can be a Gaussian distribution $\bm{\eta} \sim N(0, \Sigma)$ where $\Sigma$ can encode the similarity between words.

In the SAGE paper \cite{eisenstein2011sparse}, Laplace prior or double exponential distribution (also see the graph at \cite{laplace_prior}) is used to produce sparsity. 
Sparsity also encourages interpretability because we can now focus on fewer parameters (or words).

\bibliographystyle{plain}
\bibliography{yoshinari_notes_sage}

\end{document}

