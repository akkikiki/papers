\documentclass[11pt]{article}
\usepackage[letterpaper,margin=1in]{geometry}
\usepackage{amsbsy}
\usepackage{url}

\newcommand{\argmax}{\mathop{\rm arg~max}\limits}

\begin{document}

\section{List of Important Terminologies}
\begin{enumerate}
 \item Marginal likelihood $P(s|\alpha)$, under Bayesian framework, parameter $\theta$ being marginalized out.
 \item Posterior distribution $P(\theta,t|s,\alpha)$ 
\end{enumerate}

\section{Starting from MLE over Hidden Parse Trees}
Objective: We want to learn an optimal $\theta$, a probability distribution over rewrite rules in PCFG, given observed sentences $\boldsymbol{s}$ i.e. 
$$
\argmax_{\theta} P(\boldsymbol{s}, t|\theta)
$$
where parse trees $t$ are hidden. How do we do that? Let's start from not considering Bayesian framework. One approach is to use maximum likelihood estimation over hidden parse trees. We will try to learn $\theta$ that maximizes the marginal likelihood $P(s|\theta) = \sum_t P(s, t|\theta)$ with EM algorithm (known as ``inside-outside algorithm'').
\begin{enumerate}
 \item Predict the parse trees given $\theta$ \cite{learning_tree_annotation}.
 \item Maximize $\theta$ \cite{learning_tree_annotation}.
 \item Repeat 1. and 2. till convergence.
\end{enumerate}


\section{Applying Bayesian Framework over Hidden Parse Trees}
% TODO: clean this section up.
Then, let's considering applying Bayesian framework:
$$
P(\theta|\boldsymbol{s}) \propto P(\boldsymbol{s}|\theta) P(\theta)
$$

Use Dirichlet prior on each $\theta_A$ (where $A$ is a non-terminal):
$$
\theta_A \sim Dirichlet(\alpha_A) 
$$

As a result of including Dirichlet prior, $\theta$ now depends on a hyper-parameter $\alpha$ of a Dirichlet distribution. Thus, the posterior distribution $P(\theta|\boldsymbol{s})$ can be rewritten as \cite{cohen-johnson:2013:ACL2013}:

$$
P(\theta|\boldsymbol{s}, \alpha) \propto P(\boldsymbol{s}|\theta) P(\theta | \alpha)
$$


Assume that $s$ is distributed as $Multinomial$ i.e. $\boldsymbol{s} \sim Multinomial(\theta)$. Then, the posterior $P(\theta|\boldsymbol{s})$ or $P(\theta|\boldsymbol{s}, \alpha)$ is also distributed as $Dirichlet$:
$$
\theta|\boldsymbol{s} \sim Dirichlet(f_{\boldsymbol{t}} + \alpha)
$$
where $f_{\boldsymbol{t}}$ is the vector of production counts in $\boldsymbol{t}$ indexed by $r \in R$ \cite{johnson-griffiths-goldwater:2007:main}.

\section{Training Bayesian PCFG under supervised and unsupervised setting}

Under supervised setting, since the data consists of parse trees $\boldsymbol{t}$ \cite{cohen-johnson:2013:ACL2013}, simply replace $\boldsymbol{s}$ by $\boldsymbol{t}$:
$$
P(\boldsymbol{s}|\theta) =  P(\boldsymbol{t}|\theta) = \prod_i P(t_i | \theta)
$$

Under unsupervised setting, we regard the parse tree $t$ as a latent variable, $theta$ as a parameter. Since the data consists of sentences $\boldsymbol{s}$,
$$
P(\boldsymbol{s}|\theta) = \prod_i P(s_i | \theta) = \prod_i \sum_{t_i \in T: yield(t_i) = s_i} P(t_i | \theta)
$$

% TODO: not fully understood this part yet
%Moreover, the likelihood function becomes
%$$
%P(\boldsymbol{s}|\alpha) = \sum_t P(\boldsymbol{s}, t|\alpha) = \sum_t \int P(\boldsymbol{s}, t|\boldsymbol{\theta}) P(\boldsymbol{\theta}|\alpha) d\theta
%$$



Moreover, in unsupervised setting, we are also interested in knowing the parse trees $t$ for given sentences $s$. So the posterior $P(\theta|\boldsymbol{s}, \alpha)$ becomes:
$$
P(\theta|\boldsymbol{s}, \alpha) = \sum_{\boldsymbol{t}} P(\boldsymbol{t}, \theta | \boldsymbol{s}, \alpha) 
$$

In conclusion, under both unsupervised setting with Bayesian framework, our objective is to compute the posterior $P(\boldsymbol{t}, \theta | \boldsymbol{s}, \alpha)$ using Dirichlet prior with hyper-parameter $\alpha$.

\section{Marginal Likelihood vs. Joint likelihood}

Once again, $\theta \sim Dirichlet(\alpha)$. 
In general, marginal likelihood is a likelihood function in which some parameter variables are marginalized out. Note that marginal likelihood


when a variable $\theta$ is marginalized out.

%With labeled data $D = \{x_i, y_i\}$, the joint likelihood of input and output \cite{joint_vs_marginal} is
%$$
%\argmax_{\theta} L(\theta|X, Y) = \argmax_{\theta} \prod^n_{i = 1} P(y_i|\theta) P(x_i|y_i, \theta).
%$$
%
%With unlabeled data $D = \{x_i\}$, the marginal likelihood of input \cite{joint_vs_marginal} is
$$
\argmax_{\theta} L(\theta|X) = P(X|\theta) = \argmax_{\theta} \prod^n_{i = 1} \sum_{y \in \Omega_Y}P(y|\theta) P(x_i|y, \theta).
$$
%which need to compute every way of filling in the missing labels $y$ \cite{mlss09}.
%
%
%Marginal likelihood is used when training data is unlabeled \cite{joint_vs_marginal} (Cited slides use HMM as an example). 
%With unlabeled training data, we maximized the marginal likelihood of the input. In HMM case, the parameters $\boldsymbol{\theta}$ are transition probabilities $A$, emission probabilities $\phi$, and probability of being an initial state $\pi$ \cite{hmm_parameters}. In \cite{joint_vs_marginal}, the parameters $\boldsymbol{\theta}$ is a bit unclear so let me clarify. Transition probabilities can be computed using state sequence $\boldsymbol{y}$ alone. Therefore, $P(y_i|\theta)$ or $P(y_i|A, \pi)$ is the transition probability. Emission probabilities can be computed the observed sequence and the state sequence $\boldsymbol{x}$ and $\boldsymbol{x}$. Therefore, $P(x_i|y_i,\theta)$ or $P(x_i|y_i, \phi)$ is the emission probability. 

\subsection{Why Marginal Likelihood?}
Since outputs (= labels) are not given in latent variable models \cite{liang_jordan_klein}, we cannot compute the joint likelihood of inputs and outputs. Therefore, we would like to compute the marginal likelihood (by summing over all outputs for a given input) \cite{pcfg_marginal}. If we are using EM to learn the rule probabilities, then it attempts to maximize the marginal likehood of inputs \cite{pcfg_marginal}.

Note that marginal likelihood can appear in either Bayesian framework or non-Bayesian framework.

\section{Summary of Parameter Learning \cite{mlss09}}

Here, we assume that the prior is dirichlet distribution, likelihood is multinomial distribution, and therefore, the posterior distribution is also Dirichlet distribution.

\begin{center}
    \begin{tabular}{|l|l|l|}
    \hline
             & labeled data (=Parse trees known) & unlabeled data (= Parse trees unknown)  \\\hline
    ML       & frequency                         & EM              \\\hline
    Bayesian & updated Dirichlet distribution    & MCMC/VB         \\\hline
    \end{tabular}
\end{center}

\section{Do you understand the difference between MAP and Bayesian approach?}
``Both ML and MAP return only single and specific values for the parameter $\theta$. Bayesian estimation, by contrast, calculates fully the posterior distribution $P(\theta|X)$.'' \cite{map_vs_bayesian}.

\section{Key Features}
\begin{enumerate}
 \item PCFG
 \item Bayesian Framework
 \item Variational Bayes, Gibbs sampling, particle filter
 \item Unsupervised (parse trees not observed)
\end{enumerate}

\section{Misc. Notes}
In a supervised setting, a tree $t$ and a sentence $s$ are both observable.


\section{Future Studies}
\begin{enumerate}
 \item Read \cite{beal2003variational} and \cite{kurihara2004application}.
 \item Variational Inference, Gibbs sampling, and particle filter in general.
 \item Start from assuming that the parse trees $\boldsymbol{t}$ are observed. Then, study about the Bayesian framework applied to PCFG.
 \item Review the NLP lectures by Chris Manning on Coursera (Assignment PDFs are available at http://www.mohamedaly.info/teaching/cmp-462-spring-2013)
 \item Why marginal likelihood is important when we want to estimate the posterior using Variational Bayes?
\end{enumerate}




\bibliographystyle{plain}
\bibliography{yoshinari_notes_pcfg}

\end{document}

